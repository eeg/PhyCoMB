\section{Views}
\label{sec:views}

I'm intending this section to show some examples of the web interface a user might work with.
But it'll be time-consuming to make drawings, so maybe we'd better discuss the other things first.
So consider this section just very rough notes for now.

%--------------------------------------------------
\subsection{Drill down to results of interest}
\label{sec:views_task}
%--------------------------------------------------

Beginning of workflow in \cref{sec:workflows_task}.

\begin{enumerate}

% \label{fig:views_select_task}
    \item Select the desired \Task.
          We will focus on `state-dependent diversification' with `discrete traits' and probably `hypothesis testint,' but eventually there will be others.
% Still thinking about if we can have different possible Questions within a \Task, \eg `Is this trait associated with diversification differences?' versus `Are speciation and extinction rates estimated accurately?'

% \label{fig:views_select_method}
    \item Select the desired \Methods.
          Choose by model/technique (\eg bisse, sister clades) and/or statistics (\eg AIC, Bayes factors).
          Need a nice interface for this, but not too complicated.

% \label{fig:views_select_element}
    \item Select the desired \Elements.  A few options here:
        \begin{itemize}
            \item Those in the \Benchmark.
            \item Those in particular \Refsets.  Select from the list of \Refsets.
            \item Select based on attributes of \Elements, including attributes of \Trees and \Traits.  Will require a nice interface. % http://webworkflow.co.uk/plugins/pfSelect/
            \item All.  Warn how many this is.
        \end{itemize}
\end{enumerate}

%--------------------------------------------------
\subsection{Report for a task}
\label{sec:views_report_task}
%--------------------------------------------------

End of workflow in \cref{sec:workflows_task}.

Original vision in \cref{fig:proposal_phycomb}.
This is the single most important part of the user interface.

Interactive interface:
\begin{itemize}
    \item Select tags to float \Elements to top (within a \Refset)
    \item Same, but for categories (\ie the possible values within a column)
    \item Click arrow at top of column to sort on its categories
    \item Click arrow to fold away a \Refset % https://js-tutorials.com/demos/flexigrid_example_demo/
    \item Click something to fold away or resize columns? % https://js-tutorials.com/demos/flexigrid_example_demo/ http://www.guriddo.net/demo/bootstrap/
    \item Apply a sequence of sort/filter actions.  Each refines the previous one, rather than replacing it. % http://tablesorter.com/docs/, http://www.jeasyui.com/demo/main/index.php?plugin=DataGrid "multiple sorting", https://datatables.net/examples/basic_init/multi_col_sort.html (but without paging)
    \item Clear some/all the sort/filter actions
    \item Select \Elements (rows) to hide
    \item Hide all non-floated/non-selected \Elements
    \item Maybe color rows based on \Performance
    \item Floating headers, remains visible when scrolling
    \item Table may be large.  But show the whole thing, rather than `paging' it.  Be as space-efficient as possible.
\end{itemize}

% seems like jQuery is good for this?

Link to download CSV file (\cref{sec:downloads_report}).

%--------------------------------------------------
\subsection{Report for a method}
\label{sec:views_report_method}
%--------------------------------------------------

See \cref{sec:workflows_method}.

Most functionality like \cref{sec:views_report_task}.
But in this case, there will be a column for each \Task (instead of for each \Method).
Not sure how standard the task columns can be.
Need to think about if all tasks/questions can be answered with one number or symbol.

%--------------------------------------------------
\subsection{See element}
%--------------------------------------------------

See details of a particular \Element (\cref{sec:tables_element}).

Snazzy plot of trait on tree?

%--------------------------------------------------
\subsection{See method}
%--------------------------------------------------

See details of a particular \Method (\cref{sec:tables_method}).

%--------------------------------------------------
\subsection{Download testing files}
%--------------------------------------------------

View to choose \Elements, \Refsets, \Benchmark (\cref{sec:downloads_element}).

View to choose \Methods (\cref{sec:downloads_method}).

%--------------------------------------------------
\subsection{Contribute element}
%--------------------------------------------------

Form for a Contributor to provide the necessary information and files (\cref{sec:tables_element}).

%--------------------------------------------------
\subsection{Contribute method}
%--------------------------------------------------

Form for a Contributor to provide the necessary information and files (\cref{sec:tables_method}), and performance results.

%--------------------------------------------------
\subsection{Administrator}
%--------------------------------------------------

Need to consider extra tasks and Admin user would do.
