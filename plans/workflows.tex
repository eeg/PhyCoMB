\section{Workflows}

Here are some examples of how different kinds of users might commonly interact with \phycomb.

%--------------------------------------------------
\subsection{Viewer workflows}
\label{sec:workflows_viewer}
%--------------------------------------------------

\subsubsection{Explore performance of various methods for a particular task}
\label{sec:workflows_task}

The user's goal is to learn about how various methods perform for questions they're interested in.
This means seeing which \Methods are available for a given \Task, and their \Performance, summarized in a \Report.
This is the most important user interface component to design well.

\begin{enumerate}
    \item Select the desired \Task.
          See \cref{fig:views_select_task} for what that might look like.
    \item Select the desired \Methods.
          See \cref{fig:views_select_method} for what that might look like.
    \item Select the desired \Elements.
          See \cref{fig:views_select_element} for what that might look like.
    \item Browse the \Report.
          See \cref{sec:views_report_task} for what that might look like and the types of interactivity needed.
    \item Download the \Report.
          See \cref{sec:downloads_report}.
\end{enumerate}

\subsubsection{Explore performance of a particular method for various tasks}
\label{sec:workflows_method}

The user's goal is to learn about how a method they're interested in performs for various questions.
This means seeing which \Tasks are addressed by a given \Method, and its \Performance, summarized in a \Report.

\begin{enumerate}
    \item Select the desired \Method.
          % See \cref{fig:views_select_method} for what that might look like. % maybe panels (a) and (b)?
    \item Select the desired \Task, or default to all relevant \Tasks.
          % See \cref{fig:views_select_task} for what that might look like. % maybe panels (a) and (b)?
    \item Select the desired \Elements.
          See \cref{fig:views_select_element} for what that might look like.
    \item Browse the \Report.
          See \cref{sec:views_report_method} for what that might look like and the types of interactivity needed.
    \item Download the \Report.
          See \cref{sec:downloads_report}.
\end{enumerate}

%--------------------------------------------------
\subsection{Contributor workflows}
\label{sec:workflows_contrib}
%--------------------------------------------------

\subsubsection{Contribute a new method}

The user's goal is to add a \Method to the ones available, thinking that it will be useful for other people.
% See \cref{fig:views_contrib_method} for what some of these steps might look like.

\begin{enumerate}
    \item Browse existing \Methods to determine if it's already available.
    \item Provide a script that completely runs the \Method when provided with an \Element, returning the value for \Performance of a \Task.
          % Perhaps multiple \Performance results if multiple \Tasks are addressed?
    \item Fill in a form with basic information.
          This info helps to categorize the \Method in the database, and it will be passed along to future users who see the \Method.
        \begin{itemize}
            \item Specify what \Task it is for.
                  Maybe also what \Refset it is most interesting for, \eg if method was designed to fix a particular weakness.
            \item Explain why the new \Method is worthwhile---what was the purpose in creating it?
            \item Provide info that becomes values in columns: what category of model/technique, what kind of statistical inference, \etc (see \cref{sec:tables_model} for the required fields).
        \end{itemize}
    \item Upload \Performance results for at least the \Benchmark of \Elements.
    \item See a \Report to confirm that things look correct.
    \item Request that an Administrator accept the new \Method?
\end{enumerate}

\subsubsection{Contribute a new element}

The user's goal is to add an \Element to the ones available, thinking that it will be useful for other people.

\begin{enumerate}
    \item Browse existing \Elements to determine if it or something similar is already available.
        \begin{enumerate}
            \item If the new \Element is better than an existing one, include a deprecation request for the Administrator?
        \end{enumerate}
    \item Provide data files for the new \Element.
        \begin{enumerate}
            \item Generating script, if applicable.  Link to empirical data source. \etc
                  Needs to be completely documented and reproducible.
            \item The actual tree and trait files.
                  Or if using \Tree and/or \Trait from another \Element, link to that.
        \end{enumerate}
    \item Fill in a form with basic info:
          This info helps to categorize the \Element in the database, and it will be passed along to future users who see the \Element.
        \begin{itemize}
            \item Specify what \Task it is for, and what \Refset it will belong to.
            \item Explain why the new \Element is worthwhile---what was the purpose in creating it?
            \item Provide info that becomes values in columns: simulated/empirical, sim regime, \etc (see \cref{sec:tables_element} for the required fields). % and Tree and Trait tables
            \item Can some info be obtained automatically from the uploaded files?  \eg tree size
        \end{itemize}
    \item Upload \Performance results for all(?) \Methods that apply to that \Refset.
    \item See a \Report to confirm that things look correct.
    \item Request that an Administrator accept the new \Element.
\end{enumerate}

\subsubsection{Contribute a new performance result}
\label{sec:workflow_performance}

This could be part of the workflow in contributing a new \Method or \Element.
But it could also be done separately.
In that case, find combinations of \Method + \Element that haven't yet be run, and fill in the values.

%--------------------------------------------------
\subsection{Administrator workflows}
\label{sec:workflows_admin}
%--------------------------------------------------

\subsubsection{Alter the groupings of elements}

Alter the \Elements in a \Refset.
Alter the \Elements and/or \Refsets in a \Benchmark.

\subsubsection{Approve an addition from a Contributor}

Check the new \Method or \Element and publish it (make it available to all users) if it looks good.
Maybe a general form for other requests, \eg to deprecate an \Element?

\subsubsection{Deprecate a method or element}

Don't entirely delete from the database, but hide from most operations?

\subsubsection{Manage users}

Field requests from users who wish to become Contributors?
Or just let that be automatic, and then revoke Contributor status if someone causes trouble?
