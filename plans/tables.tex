\section{Database Tables}

I'm assuming the information would be stored in a relational database.
Here's what could be in the tables.

%--------------------------------------------------
\subsection{\Task}
%--------------------------------------------------

Unique ID: arbitrary number.
Each \Task will be identified to a User by a phrase.
There won't be many \Tasks, and perhaps only one for awhile.

Do we also need Questions or other attributes within each \Task?
For example, say my Task is state-dependent diversification.
Discrete traits versus continuous traits.
Hypothesis testing Question versus parameter estimation Question.

Questions and Tasks could be separate tables.
Each \Task could include one or more \Question.

%--------------------------------------------------
\subsection{\Refset}
%--------------------------------------------------

Unique ID: simple human-readable numbering scheme.
There will also be a phrase that identifies each one to a User.

\Task(s) it's relevant to.

Number of \Elements it contains?  Can be determined from \Elements table.

%--------------------------------------------------
\subsection{\Element}
%--------------------------------------------------

Unique ID: auto-generated as each \Element is created.
Same considerations as for labeling \Methods.
But some systematic difference for \Method versus \Element, \eg M-001 vs E-001.

\Task(s) it's relevant to.
\Refset(s) it belongs in.

\begin{itemize}
    \item \Tree
    \item \Trait, sometimes
    \item Number of items.  Determined by \Tree and \Trait, but not always in an obvious way?  Could get 50 elements via 50 trees and no traits, or via 1 tree and 50 traits, or via 50 trees and 50 traits.
    \item Name of the Contributor.
    \item Comment from the Contributor.
    \item Date contributed.
\end{itemize}

%--------------------------------------------------
\subsection{\Tree}
%--------------------------------------------------

Unique ID: arbitrary number.

Each \Tree is actually a set of trees, but with the same properties.

If the trees are generated by a script, do we need to store them?
Probably safer to do so.  But could get unwieldy.
External files?  Could be good if they'll commonly be downloaded anyway.

If the same trees are used in multiple \Elements (\ie with different \Traits), it would be nice to use the same \Tree.

\begin{itemize}
    \item Number of tips: single number or range
    \item Source: simulated, empirical
    \item Perhaps some other columns
    \item Tags: perhaps multiple ones, allowed values will be figured out over time (separate table)
\end{itemize}

%--------------------------------------------------
\subsection{\Trait}
%--------------------------------------------------

Unique ID: arbitrary number.

If the traits are generated by a script, do we need to store them?
Also external files, like or with trees?

Columns:
\begin{itemize}
    \item Value type: discrete, continuous
    \item Perhaps some others
\end{itemize}

Also tags.  Different ones than for \Trees.

%--------------------------------------------------
\subsection{\Method}
%--------------------------------------------------

Unique ID: generate as each \Method is created.
Users will see that identifier, so should be human-readable.
Could be arbitrary numbers or letters, either generated sequentially or randomly (to reduce user bias).
Or could reflect the content of the \Method, \eg first letter of model/technique and statistics.

\Task(s) it's relevant to.
Or only \Refset(s) it's relevant to, and then that is mapped to \Task(s)?

Model or Technique.
Statistical approach.
(These should be categories.  Maybe their own Table?  With links to papers/citations?)

Script.
Should take \Element as input, return a simple answer to \Task.

    \item Name of the Contributor.
    \item Comment from the Contributor.
    \item Date contributed.

%--------------------------------------------------
\subsection{\Performance}
%--------------------------------------------------

Unique ID: arbitrary number.

Each combination of \Method and \Task (or at least Question) has a unique \Performance result.
Hopefully this can be just a single number, \eg proportion of trees on which the conclusion was correct.

But we might need more, \eg accuracy of estimates for various parameters.
This could depend on both the \Method and the \Question.

%--------------------------------------------------
% "Normalization rules" for relational database structure:
% https://en.wikipedia.org/wiki/Database_normalization
% 
% 1a.  Every cell contains a single value, not a list of values.
% 1b.  No repeating group of columns (like item1, item2...).  Instead, create another table with one-to-many.
% 
% 2a.  Every non-key column is fully dependent on the primary key.
% 2b   And if the primary key is made up of several columns, every non-key column depends on the entire key.
% 
% 3a.  The non-key columns don't depend on each other.
% 3b.  They depend on the (entire) primary key (rule 2), not other non-key columns.
% 
% There are more normalization rules.  But the basic idea is to think through operations and avoid potential mistakes:
%   * Adding data:   Does it need to be added in more than one place?
%   * Changing data: Could it accidentally not be changed in all places?
%   * Deleting data: Is additional information unintentionally lost?
% 
% A normalization rule could broken to improve performance, or there can be weird situations.  But breaking a rule should be intentional, well-documented, and with extra programming logic to handle it with care.
%--------------------------------------------------
