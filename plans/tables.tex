\section{Database Tables}
\label{sec:tables}

I'm assuming the information would be stored in a relational database.
Below is what could be in the tables.

I think it's realistic to settle on the set of tables and their relationships early on.
But some of the content in the tables (columns, allowable values for some columns) will need to be adjusted as we go along.

%--------------------------------------------------
\subsection{Elements}
\label{sec:tables_element}
%--------------------------------------------------

Each \Element consists of one \Tree (\cref{sec:tables_tree}), optionally one \Trait (\cref{sec:tables_trait}), and some other information.
% Reason for \Tree and \Trait being separate entities, rather than only parts of \Element, is so they can be reused, \eg same \Tree with different \Traits makes for different \Elements.

\begin{description}
    \item[Unique ID] Arbitrary, \eg E-47295.  Auto-generated when created.
    \item[Tree] Link to one \Tree.
    \item[Trait] Link to one \Trait, or empty.
    \item[Reference set] Link to one or more \Refsets (\cref{sec:tables_refset}) that include this \Element.
    \item[Benchmark] Link to \Benchmarks (\cref{sec:tables_benchmark}) that include this \Element, if any.
    \item[Number of items] Positive integer.
            It's determined by the combination of \Tree and \Trait, but the Contributor will provide this info.
            % Could get 50 elements via 50 trees and no traits, or via 1 tree and 50 traits, or via 50 trees and 50 traits.
    \item[Correct outcomes] What answer should be obtained.
            This will be compared against the output of \Methods.
            Each \Task will need to have a correct outcome defined.
    \item[Contribution info] (could be same or different from info for \Tree and/or \Trait)
        \begin{description}
            \item[Contributor] a registered user (\cref{sec:users_contributor})
            \item[Date] auto-populated when \Element is created
            \item[Comment] provided by the Contributor: 50 character summary, then a few lines of free text
        \end{description}
\end{description}

It will be common for a user to download one or more \Elements (\cref{sec:downloads_element}).

%--------------------------------------------------
\subsection{Trees}
\label{sec:tables_tree}
%--------------------------------------------------

Each \Tree object is actually a set of trees, all with the same properties.
Here are those properties:

\begin{description}
    \item[Unique ID] Arbitrary, \eg T-83247.  Auto-generated when created.
    \item[The trees] Each individual tree is itself stored as a \href{http://evolution.genetics.washington.edu/phylip/newicktree.html}{Newick string}.
            Those strings could reside directly within the database; they can be quite long, though, which might be troublesome.
            Or the database entry could be a link to text file(s) containing the trees; this might be better because such files will frequently be downloaded by users (\cref{sec:downloads_element}).
    \item[Generating script] The code used to simulate or otherwise create the trees.
            A file to download.  Not all \Trees will have one.
    \item[Number of trees] Positive integer.
    \item[Contribution info]:
        \begin{description}
            \item[Contributor] a registered user (\cref{sec:users_contributor})
            \item[Date] auto-populated when \Tree is created
            \item[Comment] provided by the Contributor: 50 character summary, then a few lines of free text
        \end{description}
    \item[Elements] The \Element(s) to which the \Tree belongs.
    \item[Traits] The \Trait(s) that use this \Tree.  (Not sure if we need this.)
    \item[Columns of tree info] Will be figured out as we go along, but likely ones are:
        \begin{description}
            \item [Number of tips] single number or numeric range
            \item [Source] `simulated' or `empirical'
        \end{description}
    \item [Tags] Various descriptive words.
        The idea with tags is that they are not necessarily alternatives (like `simulated' versus `empirical'), and there could be any number per \Tree.
        With use, we might realize that some tags can be converted to columns, and maybe vice versa.
        Tags probably involves two extra database tables:
        (1) columns are TagID and TagName, one row per tag;
        (2) columns are TreeID and TagID, one row per tag per tree.
\end{description}

%--------------------------------------------------
\subsection{Traits}
\label{sec:tables_trait}
%--------------------------------------------------

Each \Trait object consists of at least one trait value per species.
There could be multiple such sets in one \Trait object, \eg when simulating data so that each tree in a \Tree goes with one trait set in a \Trait.

A \Trait isn't useful on its own---it only makes sense when associated with a \Tree.
A \Trait will usually only correspond to one \Tree, but that's not required.
For example, if empirical data are used to make different \Trees with different phylogenetic inference methods, we would have many \Trees and one \Trait.

All the trait sets within a \Trait have the same properties.
Here are those properties:

\begin{description}
    \item[Unique ID] Arbitrary, \eg A-57387.  Auto-generated when created.
    \item[The traits] Each set of traits is simply a list of numbers, labeled by tip/species name.
            As for \Trees (\cref{sec:tables_tree}), this info could reside directly within the database or in a linked text file (\eg CSV), which will frequently be downloaded by users (\cref{sec:downloads_element}).
    \item[Generating script] The code used to simulate or otherwise create the traits.
            A file to download.  Not all \Traits will have one.
            Might be the same as the generating script for the corresponding \Tree.
    \item[Number of trait sets] Positive integer.  Probably either 1, or one per tree.
    \item[Contribution info] (might or might not be identical to the corresponding \Tree)
        \begin{description}
            \item[Contributor] a registered user (\cref{sec:users_contributor})
            \item[Date] auto-populated when \Trait is created
            \item[Comment] provided by the Contributor: 50 character summary, then a few lines of free text
        \end{description}
    \item[Trees] The \Tree(s) to which the \Trait corresponds.
    \item[Columns of trait info] Will be figured out as we go along, but likely ones are:
        \begin{description}
            \item [Numerical type] `discrete' or `continuous'
            \item [Source] `simulated' or `empirical'
        \end{description}
    \item [Tags] Various descriptive words.
            (See tag notes in \cref{sec:tables_tree}.  Different tags for \Trees and \Traits, though.)
\end{description}

%--------------------------------------------------
\subsection{Reference Set}
\label{sec:tables_refset}
%--------------------------------------------------

Each \Refset is a collection of \Elements (perhaps hundreds), plus some other information.

\begin{description}
    \item[Unique ID] Arbitrary, \eg R-43853.  Auto-generated when created.
    \item[Name] A short phrase that identifies the \Refset to a human.  Could be instead of the Unique ID.
    \item[Description] An explanation of what this \Refset is designed to test.
    \item[Elements] Link to \Element(s) in the \Refset.  It should also be easy to obtain the total number of \Elements.
    \item[Benchmarks] Link to \Benchmark(s) that contain parts of this \Refset, if any.
    \item[Methods] Link to \Method(s) for which this \Refset is relevant.
    \item[History] Notes from Contributors who have created or changed the \Refset.  (More than one column, but I'm not sure of the best format.)
\end{description}

(I'm not sure how much linking across tables is necessary.
For example, I've included links to \Benchmark and \Method here, but perhaps those connections could be obtained merely from the \Benchmark and \Method tables themselves.)

%--------------------------------------------------
\subsection{Benchmark Set}
\label{sec:tables_benchmark}
%--------------------------------------------------

Each \Benchmark is a collection of \Elements (perhaps dozens), linked to one specific \Task, plus some other information.
These may be updated frequently, as new \Elements are added and old ones are removed.

\begin{description}
    \item[Unique ID] There won't be many of them, so we could have more meaningful names.  These could be constructed from the name of the \Task and a number (in case we want to be trying out a few benchmark sets at once).
    \item[Name] A short phrase that identifies the \Benchmark to a human.
    \item[Description] An explanation of what this \Benchmark is designed to test.
    \item[Task] Link to the \Task.
    \item[Elements] Link to \Element(s) in the \Benchmark.  It should also be easy to obtain the total number of \Elements and their \Refset membership.
    \item[Methods] Link to \Method(s) for which this \Benchmark is relevant.
    \item[Correct outcome] What answer should be obtained by an effective \Method, for a particular \Task.  This will be the same for all \Elements in the \Benchmark.
    \item[History] Notes from Contributors who have created or changed the \Benchmark.  (More than one column, but I'm not sure of the best format.)
\end{description}

%--------------------------------------------------
\subsection{Task}
\label{sec:tables_task}
%--------------------------------------------------

The top layer of organization is the \Task.
There won't be many, and perhaps only one for awhile (\ie state-dependent diversification).

Not discussed in the original proposal (\cref{sec:proposal}) is the concept of grouping tasks or questions within them.
For example, if the task is state-dependent diversification, there could be different sub-tasks for discrete-valued and continuous-valued traits.
We might also want different questions for hypothesis testing (Is my trait associated with diversification shifts?) versus parameter estimation (How much higher are speciation rates for this state?).
This latter level of detail is essential for reporting results simply (\cref{sec:tables_result}).

I think designing this layer of organization well will be quite important for \phycomb to be useful.
But we may not know the best strategy until we see how it grows.
So we'll need more discussions and probably some flexibility here.
For the moment, let's assume that \Task is defined as specifically as necessary.

%--------------------------------------------------
\subsection{Methods}
\label{sec:tables_method}
%--------------------------------------------------

\cref{sec:tables_element,sec:tables_tree,sec:tables_trait,sec:tables_refset,sec:tables_benchmark,sec:tables_task} were all about the testing datasets themselves.
Now we consider the analysis methods that the tests are designed to evaluate.

\begin{description}
    \item[Unique ID] Arbitrary, \eg M-93925.  Auto-generated when created.
    \item[Name] A few words describing the approach. % (Or not?  Or should \Trees, \Traits, \Elements, \etc all have this?)
    \item[Analysis script] The code used to run the method.  A file to download.
    \item[Contribution info]:
        \begin{description}
            \item[Contributor] a registered user (\cref{sec:users_contributor})
            \item[Date] auto-populated when \Method is created
            \item[Comment] provided by the Contributor: 50 character summary, then a few lines of free text
        \end{description}
    \item[Scope] Some indication of what this method can be used on, but not sure exactly what.
            Could be a list of \Refsets.  Definitely needs to map to \Tasks, either directly or indirectly.
    \item[Columns of methods info] Will be figured out as we go along, but likely ones are:
        \begin{description}
            \item [Technique] `model-based' or `descriptive' (need better terminology)
            \item [Statistics] `Bayes factors', `AIC', `parameter values', `model averaging', `sign test', \etc
        \end{description}
    \item [Tags] Various descriptive words. % e.g., sse, sister clades
            (See tag notes in \cref{sec:tables_tree}.  Different tags for \Methods, though.)
\end{description}

Methods often come in groups, and I'm not sure how to handle this.
For example, the same basic procedure could be tweaked in a few ways (in the algorithm, or the type of stats used), and testing would reveal which was best.
Some possible approaches:
\begin{itemize}
    \item Ignore this complication.
          Each \Method is a stand-alone entity, and Contributors can just explain in the Comments if they want.
    \item Ignore this complication in the database structure, but maintain by hand a high-level summary of the available \Methods.
          Could be tough if \Methods are added frequently, though could ask Contributor for suggestions on how to update the summary accordingly.
    \item Collect enough meta-data (Columns, Tags, \etc) to generate a summary of relationships among \Methods.
          Not clear that this can usefully be done algorithmically, though.
    \item Allow one \Method to link to others.
          But then get into issues of how similar is similar enough to link (\eg `coded like' versus `inspired by').
          And then there would be lots of code reuse among scripts of different-but-related \Methods.
    \item Allow multiple analysis scripts and/or libraries of shared functions per \Method.
          Seems likely to get messy, though, especially when summarizing results.
    \item Allow one explicit level of hierarchy in \Methods.
          Contributor groups them together if they share code, but each is numbered separately and results are reported separately for each variant.
\end{itemize}

I'm currently leaning toward the by-hand summary or the single layer of explicit hierarchy, but other thoughts would be very welcome.

%--------------------------------------------------
\subsection{Results}
\label{sec:tables_result}
%--------------------------------------------------

When a \Method is run on an \Element, the output should be a clear answer to a \Task.
This answer is a single \Result.
Probably it will be a number (\eg proportion of trees that yielded an answer of `yes'; overall difference in parameter estimates), but perhaps it could be a binary or categorical answer.

The table of \Results would then have these columns:
\begin{description}
    \item[Unique ID] of the \Result
    \item[Method] Link to \Method
    \item[Element] Link to \Element
    \item[Task] Link to \Task
    \item[Value] The answer
    \item[Detail] Maybe we should store the answer on each \Tree/\Trait within the \Element?
          Potentially useful to have for an end user, but maybe too much detail for \phycomb to interpret.
\end{description}

It is possible that we don't need to include \Task here, if \Method is already specific enough that it applies to a single \Task.
See discussion elsewhere about the specificity of \Methods (\cref{sec:tables_method}) and \Tasks (\cref{sec:tables_task}).

One concern I have is that reducing each \Result to a single number would cause analyses to be simple-minded rather than nuanced.
On the other hand, we need to keep things simple enough that we can summarize across lots of tests and methods.
Would be good to discuss this.

We will definitely need nice ways to summarize this huge table of results for users.
These summaries are called \Reports elsewhere in this document.
But I'm not sure whether this would be done within the database structure or when generating views (\eg \cref{sec:views_report_task,sec:views_report_method}).
Some summaries we'll want:
\begin{itemize}
    \item Overall performance of a \Method on a \Refset for a \Task
    \item Overall performance of a \Method on a \Benchmark for a \Task
    \item Overall performance of a \Method at a \Task
    \item Performance of all \Methods on an \Element
    \item Performance of all \Methods on an \Refset
    \item Performance of all \Methods on an \Benchmark
\end{itemize}




%--------------------------------------------------
% "Normalization rules" for relational database structure:
% https://en.wikipedia.org/wiki/Database_normalization
% 
% 1a.  Every cell contains a single value, not a list of values.
% 1b.  No repeating group of columns (like item1, item2...).  Instead, create another table with one-to-many.
% 
% 2a.  Every non-key column is fully dependent on the primary key.
% 2b   And if the primary key is made up of several columns, every non-key column depends on the entire key.
% 
% 3a.  The non-key columns don't depend on each other.
% 3b.  They depend on the (entire) primary key (rule 2), not other non-key columns.
% 
% There are more normalization rules.  But the basic idea is to think through operations and avoid potential mistakes:
%   * Adding data:   Does it need to be added in more than one place?
%   * Changing data: Could it accidentally not be changed in all places?
%   * Deleting data: Is additional information unintentionally lost?
% 
% A normalization rule could broken to improve performance, or there can be weird situations.  But breaking a rule should be intentional, well-documented, and with extra programming logic to handle it with care.
%--------------------------------------------------
