\section{Database Tables}

I'm assuming the information would be stored in a relational database.
Here's what could be in the tables.

I think it's realistic to settle on the set of tables and their relationships early on.
But some of the content in the tables (columns, allowable values for some columns) will probably be adjusted as we go along.

%--------------------------------------------------
\subsection{\Task}
%--------------------------------------------------

Unique ID: arbitrary number.
Each \Task will be identified to a User by a phrase.
There won't be many \Tasks, and perhaps only one for awhile.

Do we also need Questions or other attributes within each \Task?
For example, say my Task is state-dependent diversification.
Discrete traits versus continuous traits.
Hypothesis testing Question versus parameter estimation Question.

Questions and Tasks could be separate tables.
Each \Task could include one or more Question.

%--------------------------------------------------
\subsection{\Refset}
%--------------------------------------------------

Unique ID: simple human-readable numbering scheme.
There will also be a phrase that identifies each one to a User.

\Task(s) it's relevant to.

Number of \Elements it contains?  Can be determined from \Elements table.

%--------------------------------------------------
\subsection{\Benchmark}
%--------------------------------------------------

some \Elements, which bring along their \Refset membership

%--------------------------------------------------
\subsection{Elements}
\label{sec:tables_elements}
%--------------------------------------------------

Each \Element consists of one \Tree (\cref{sec:tables_trees}), optionally one \Trait (\cref{sec:tables_traits}), and some other information.
% Reason for \Tree and \Trait being separate entities, rather than only parts of \Element, is so they can be reused, \eg same \Tree with different \Traits makes for different \Elements.

\begin{description}
    \item[Unique ID] Arbitrary, \eg E-47295.  Auto-generated when created.
    \item[Tree] Link to corresponding \Tree.
    \item[Trait] Link to corresponding \Trait, if any.
    \item[Refset] Link to one or more \Refsets for which this \Element is relevant.
    \item[Number of items] Positive integer.
            It's determined by the combination of \Tree and \Trait, but the Contributor will provide this info.
            % Could get 50 elements via 50 trees and no traits, or via 1 tree and 50 traits, or via 50 trees and 50 traits.
    \item[Contribution info] (could be same or different from info for \Tree and/or \Trait)
        \begin{description}
            \item[Contributor] a registered user, see \cref{sec:users_contributor}
            \item[Date] auto-populated when \Element is created
            \item[Comment] a few sentences provided by the Contributor
        \end{description}
\end{description}

It will be common for a user to download one or more \Elements (\cref{sec:downloads_element}).

%--------------------------------------------------
\subsection{Trees}
\label{sec:tables_trees}
%--------------------------------------------------

Each \Tree object is actually a set of trees, all with the same properties.
Here are those properties:

\begin{description}
    \item[Unique ID] Arbitrary, \eg T-83247.  Auto-generated when created.
    \item[The trees] Each individual tree is itself stored as a \href{http://evolution.genetics.washington.edu/phylip/newicktree.html}{Newick string}.
            Those strings could reside directly within the database; they can be quite long, though, which might be troublesome.
            Or the database entry could be a link to text file(s) containing the trees; this might be better because such files will frequently be downloaded by users (\cref{sec:downloads_element}).
    \item[Generating script] The code used to simulate or otherwise create the trees.
            A file to download.  Not all \Trees will have one.
    \item[Number of trees] Positive integer.
    \item[Contribution info]:
        \begin{description}
            \item[Contributor] a registered user, see \cref{sec:users_contributor}
            \item[Date] auto-populated when \Tree is created
            \item[Comment] a few sentences provided by the Contributor
        \end{description}
    \item[Elements] The \Element(s) to which the \Tree belongs.
    \item[Columns of tree info] Will be figured out as we go along, but likely ones are:
        \begin{description}
            \item [Number of tips] single number or numeric range
            \item [Source] `simulated' or `empirical'
        \end{description}
    \item [Tags] Various descriptive words.
        The idea with tags is that they are not necessarily alternatives (like `simulated' versus `empirical'), and there could be any number per \Tree.
        With use, we might realize that some tags can be converted to columns, and maybe vice versa.
        Tags probably involves two extra database tables:
        (1) columns are TagID and TagName, one row per tag;
        (2) columns are TreeID and TagID, one row per tag per tree.
\end{description}

%--------------------------------------------------
\subsection{Traits}
\label{sec:tables_traits}
%--------------------------------------------------

Each \Trait object consists of at least one trait value per species.
There could be multiple such sets in one \Trait object.
In that case, all the trait sets have the same properties.
Here are those properties:

\begin{description}
    \item[Unique ID] Arbitrary, \eg A-57387.  Auto-generated when created.
    \item[The traits] Each set of traits is simply a list of numbers, labeled by tip/species name.
            As for \Trees (\cref{sec:tables_trees}), this info could reside directly within the database or in a linked text file (\eg CSV), which will frequently be downloaded by users (\cref{sec:downloads_element}).
    \item[Generating script] The code used to simulate or otherwise create the traits.
            A file to download.  Not all \Traits will have one.
            Might be the same as the generating script for the corresponding \Tree.
    \item[Number of trait sets] Positive integer.
    \item[Contribution info] (might or might not be identical to the corresponding \Tree(s))
        \begin{description}
            \item[Contributor] a registered user, see \cref{sec:users_contributor}
            \item[Date] auto-populated when \Trait is created
            \item[Comment] a few sentences provided by the Contributor
        \end{description}
    \item[Trees] The \Tree(s) to which the \Trait corresponds.
    \item[Columns of trait info] Will be figured out as we go along, but likely ones are:
        \begin{description}
            \item [Numerical type] `discrete' or `continuous'
            \item [Source] `simulated' or `empirical'
        \end{description}
    \item [Tags] Various descriptive words.
            (See tag notes in \cref{sec:tables_trees}.  Different tags for \Trees and \Traits, though.)
\end{description}

%--------------------------------------------------
\subsection{\Method}
%--------------------------------------------------

Unique ID: generate as each \Method is created.
Users will see that identifier, so should be human-readable.
Could be arbitrary numbers or letters, either generated sequentially or randomly (to reduce user bias).
Or could reflect the content of the \Method, \eg first letter of model/technique and statistics.

\Task(s) it's relevant to.
Or only \Refset(s) it's relevant to, and then that is mapped to \Task(s)?

Model (bisse, etc) or Technique (sister clades).
Statistical approach (bayes factors, aic, model averaging, sign test, etc).
(These should be categories.  Maybe their own Table?  With links to papers/citations?)

Script.
Should take \Element as input, return a simple answer to \Task.

How to run a single analysis and return results in multiple forms, \eg hypothesis testing question and parameter estimation question?
Call those separate \Methods?  Link those \Methods for clarity and code reuse?

\begin{itemize}
    \item Name of the Contributor.
    \item Comment from the Contributor.
    \item Date contributed.
\end{itemize}

%--------------------------------------------------
\subsection{\Performance}
%--------------------------------------------------

Unique ID: arbitrary number.

Each combination of \Method and \Task (or at least Question) has a unique \Performance result.
Hopefully this can be just a single number, \eg proportion of trees on which the conclusion was correct.

But we might need more, \eg accuracy of estimates for various parameters.
This could depend on both the \Method and the Question.

%--------------------------------------------------
% "Normalization rules" for relational database structure:
% https://en.wikipedia.org/wiki/Database_normalization
% 
% 1a.  Every cell contains a single value, not a list of values.
% 1b.  No repeating group of columns (like item1, item2...).  Instead, create another table with one-to-many.
% 
% 2a.  Every non-key column is fully dependent on the primary key.
% 2b   And if the primary key is made up of several columns, every non-key column depends on the entire key.
% 
% 3a.  The non-key columns don't depend on each other.
% 3b.  They depend on the (entire) primary key (rule 2), not other non-key columns.
% 
% There are more normalization rules.  But the basic idea is to think through operations and avoid potential mistakes:
%   * Adding data:   Does it need to be added in more than one place?
%   * Changing data: Could it accidentally not be changed in all places?
%   * Deleting data: Is additional information unintentionally lost?
% 
% A normalization rule could broken to improve performance, or there can be weird situations.  But breaking a rule should be intentional, well-documented, and with extra programming logic to handle it with care.
%--------------------------------------------------
